\documentclass[11pt]{article}

\usepackage{lmodern}

\usepackage{xcolor}
\usepackage{listings}
\lstdefinestyle{BashInputStyle}{
  language=sh,
  basicstyle=\footnotesize\ttfamily,
  showstringspaces=false,
  %numbers=left,
  numberstyle=\tiny,
  numbersep=3pt,
  frame=tb,
  columns=fullflexible,
  backgroundcolor=\color{yellow!20},
  linewidth=0.9\linewidth,
  xleftmargin=0.1\linewidth
}

\newcommand{\path}[1]{{\small\bfseries\ttfamily{#1}}}
\newcommand{\shell}[1]{{\small\bfseries\ttfamily{#1}}}

\begin{document}
\lstset{language=bash}
\lstset{style=BashInputStyle}

\section{Initial Setup}

This part applies to nims2, since that's the one that I set up
from scratch.
Plugged ethernet cable into port nearest the edge, and registered
on the PNNL network by trying to get to google.com using Firefox. 
In the network control panel, set the
dhcp client ID to ``nims2'' so PNNL's DNS server picks it up.

Setting the hostname requires changing it in \path{/etc/hosts}
\begin{lstlisting}
    sudo echo nims2 > /etc/hostname
\end{lstlisting}
and also editing \path{/etc/hosts} to point 127.0.1.1 at nims2:
\begin{lstlisting}
    127.0.0.1	localhost
    127.0.1.1	nims1
\end{lstlisting}

I changed the editor from \shell{nano} to \shell{vim.tiny}, as follows. 
This affects commands like \shell{visudo} and \shell{vipw}, after I
screwed up the \path{/etc/sudoers} file by not realizing that it
had invoked \shell{nano}.
\begin{lstlisting}
sudo update-alternatives --config editor
There are 3 choices for the alternative editor (providing /usr/bin/editor).

  Selection    Path               Priority   Status
------------------------------------------------------------
* 0            /bin/nano           40        auto mode
  1            /bin/ed            -100       manual mode
  2            /bin/nano           40        manual mode
  3            /usr/bin/vim.tiny   10        manual mode
\end{lstlisting}

For remote access, we also need to install the SSH server:
\begin{lstlisting}
    apt-get install openssh-server
\end{lstlisting}
At this point, it's basically functional on the network, and
accessible as a headless server.

\section{Users and Groups}

I added a \shell{nims} user and \shell{amaxwell} user,
for the NIMS executables and my user, respectively, using
\lstinline$adduser foo$ 
(which creates home directories and
sets up permissions appropriately). The \shell{nims} user's
password is the same as that of \shell{owner}.
I then set
my user up with \shell{sudo} access via
\lstinline$sudo visudo$, 
which invokes the \shell{vi} editor.
After adding the line 
\begin{lstlisting}
    amaxwell ALL=(ALL) ALL
\end{lstlisting}
at the end of the file, \shell{amaxwell} can execute commands
as root.

I edited \path{/etc/passwd} manually to change UID and GID
of \shell{nims} as follows:
\begin{lstlisting}
    nims:x:200:200:NIMS user,,,:/home/nims:/bin/bash
\end{lstlisting}
Next, I edited \path{/etc/group} manually to change the
\shell{nims} group ID to 200 (this group was automatically
created by \lstinline$adduser nims$):
\begin{lstlisting}
    nims:x:200:
\end{lstlisting}
This signifies that \shell{nims} is a system user, and prepares
us for future usage of groups for filesystem permissions. Finally,
\begin{lstlisting}
    sudo chown nims:nims /home/nims
\end{lstlisting}
to fix the mess we just made with user and group ID.

\section{Configuration}

To set timezone:
\begin{lstlisting}
    dpkg-reconfigure tzdata
\end{lstlisting}
and follow the menu prompts. I set this to Los\_Angeles, which is
the tzdata name for Pacific time. Next, to install NTP support:
\begin{lstlisting}
    apt-get install system-config-date
    apt-get install ntp
    sudo system-config-date
\end{lstlisting}
Edit \path{/etc/ntp.conf} to include
\begin{lstlisting}
    server time.apple.com iburst
    server 130.20.248.2 iburst prefer
    server 130.20.128.83 iburst prefer
\end{lstlisting}
since our network blocks outside NTP servers. Note that this will
likely need to change for deployment; ideally, we'd have a GPS
receiver onboard. Reboot or restart the ntp service after changing
the config file.

\section{Startup}
The following script in \path{/etc/init.d} is used to
start the NIMS binaries.
\begin{lstlisting}
    #!/bin/sh

    # System Startup for NIMS
    # to be installed in /etc/init.d

    APPLICATION_PATH=/home/nims/bin

    killproc() {
            pid=`pidof $1`
            [ "$pid" != "" ] && kill -9 $pid
    }

    case $1 in

       start)
          echo "Starting NIMS applications."
          if [ -e $APPLICATION_PATH/nims ]; then
              pushd .
              cd $APPLICATION_PATH
              $APPLICATION_PATH/nims &
              popd
          fi
          ;; 

       stop)
          echo "Stopping NIMS (Normal Mode)"
          killall nims
          ;;

       *)
         echo "Usage: nims.sh  {start|stop}"
         exit 1
         ;;
    esac

    exit 0
\end{lstlisting}
To activate it, we use
\begin{lstlisting}
    sudo service nims start|stop
\end{lstlisting}
from the command prompt. Eventually we'll set it to run at boot with
\begin{lstlisting}
    # set to start in runlevels [2-5] at S99, stop in [0,1,6] at K1
    # was using S99 only in testing (K number = 100 - S number)
    /usr/sbin/update-rc.d nims start 99 2 3 4 5 . stop 1 0 1 6 .
\end{lstlisting}
or similar.

\end{document}