\documentclass[11pt]{article}

\usepackage{lmodern}

\usepackage{xcolor}
\usepackage{listings}
\lstdefinestyle{BashInputStyle}{
  language=sh,
  basicstyle=\footnotesize\ttfamily,
  showstringspaces=false,
  %numbers=left,
  numberstyle=\tiny,
  numbersep=3pt,
  frame=tb,
  columns=fullflexible,
  backgroundcolor=\color{yellow!20},
  linewidth=0.9\linewidth,
  xleftmargin=0.1\linewidth
}

\usepackage{dirtree}

\newcommand{\path}[1]{{\small\bfseries\ttfamily{#1}}}
\newcommand{\shell}[1]{{\small\bfseries\ttfamily{#1}}}

\begin{document}
\lstset{language=bash}
\lstset{style=BashInputStyle}

\section{Initial Setup}

This part applies to nims2, since that's the one that I set up
from scratch.
Plugged ethernet cable into port nearest the edge, and registered
on the PNNL network by trying to get to google.com using Firefox. 
In the network control panel, set the
dhcp client ID to ``nims2'' so PNNL's DNS server picks it up.

Setting the hostname requires changing it in \path{/etc/hosts}
\begin{lstlisting}
    sudo echo nims2 > /etc/hostname
\end{lstlisting}
and also editing \path{/etc/hosts} to point 127.0.1.1 at nims2:
\begin{lstlisting}
    127.0.0.1	localhost
    127.0.1.1	nims1
\end{lstlisting}

I changed the editor from \shell{nano} to \shell{vim.tiny}, as follows. 
This affects commands like \shell{visudo} and \shell{vipw}, after I
screwed up the \path{/etc/sudoers} file by not realizing that it
had invoked \shell{nano}.
\begin{lstlisting}
sudo update-alternatives --config editor
There are 3 choices for the alternative editor (providing /usr/bin/editor).

  Selection    Path               Priority   Status
------------------------------------------------------------
* 0            /bin/nano           40        auto mode
  1            /bin/ed            -100       manual mode
  2            /bin/nano           40        manual mode
  3            /usr/bin/vim.tiny   10        manual mode
\end{lstlisting}

For remote access, we also need to install the SSH server:
\begin{lstlisting}
    apt-get install openssh-server
\end{lstlisting}
At this point, it's basically functional on the network, and
accessible as a headless server.

\section{Users and Groups}

I added a \shell{nims} user and \shell{amaxwell} user,
for the NIMS executables and my user, respectively, using
\lstinline$adduser foo$ 
(which creates home directories and
sets up permissions appropriately). The \shell{nims} user's
password is the same as that of \shell{owner}.
I then set
my user up with \shell{sudo} access via
\lstinline$sudo visudo$, 
which invokes the \shell{vi} editor.
After adding the line 
\begin{lstlisting}
    amaxwell ALL=(ALL) ALL
\end{lstlisting}
at the end of the file, \shell{amaxwell} can execute commands
as root.

I edited \path{/etc/passwd} manually to change UID and GID
of \shell{nims} as follows:
\begin{lstlisting}
    nims:x:200:200:NIMS user,,,:/home/nims:/bin/bash
\end{lstlisting}
Next, I edited \path{/etc/group} manually to change the
\shell{nims} group ID to 200 (this group was automatically
created by \lstinline$adduser nims$):
\begin{lstlisting}
    nims:x:200:
\end{lstlisting}
This signifies that \shell{nims} is a system user, and prepares
us for future usage of groups for filesystem permissions. Finally,
\begin{lstlisting}
    sudo chown nims:nims /home/nims
\end{lstlisting}
to fix the mess we just made with user and group ID.

\section{Configuration}

To set timezone:
\begin{lstlisting}
    dpkg-reconfigure tzdata
\end{lstlisting}
and follow the menu prompts. I set this to Los\_Angeles, which is
the tzdata name for Pacific time. Next, to install NTP support:
\begin{lstlisting}
    apt-get install system-config-date
    apt-get install ntp
    sudo system-config-date
\end{lstlisting}
Edit \path{/etc/ntp.conf} to include
\begin{lstlisting}
    server time.apple.com iburst
    server 130.20.248.2 iburst prefer
    server 130.20.128.83 iburst prefer
\end{lstlisting}
since our network blocks outside NTP servers. Note that this will
likely need to change for deployment; ideally, we'd have a GPS
receiver onboard. Reboot or restart the ntp service after changing
the config file.

\subsection{Dependencies}
The UW webapp and Echometrics have some dependencies, but they
can be installed using the Mint package manager, with one exception.
\begin{lstlisting}
    sudo apt-get install python-scipy
    sudo apt-get install python-pandas
    sudo apt-get install python-matplotlib
    sudo apt-get install python-tornado
    sudo apt-get install chromium-browser
    sudo apt-get install python2.7-dev
\end{lstlisting}
The exception is the \shell{posix\_ipc} package, which is not
provided by Ubuntu. In that case, we can install \shell{python-pip},
which is provided by Ubuntu, and install \shell{posix\_ipc} from
PyPi as follows:
\begin{lstlisting}
    sudo apt-get install python-pip
    sudo pip install posix_ipc
\end{lstlisting}
This avoids manual download and configuration/compile, and pip
can also be used for later updates.

\section{Installation and Startup}
The \path{install-nims.sh} script in the Subversion repository can
be used to install all of the PNNL binaries and the webapp via \shell{scp}
(requires you to install the \shell{sshpass} program on your build
machine). It requires a single
argument, which is the password for the \shell{nims} account, and
installs programs as follows:

\dirtree{%
.1 /.
.2 home.
.3 nims.
.4 bin.
.5 config.yaml\DTcomment{configuration file for NIMS binaries}.
.5 detector.
.5 ingester.
.5 nims.
.5 nims-init\DTcomment{init script; install at \path{/etc/init.d/nims}}.
.5 tracker.
.5 webapp.
.6 nims.py\DTcomment{the main program; other files live alongside}.
}

\subsection{Startup}
The init script must be copied manually, but it only changes if
the path to the \shell{nims} binary changes or the path to the
webapp (\shell{nims.py}) changes. To install the script,
\begin{lstlisting}
    cd ~nims/bin
    sudo cp nims-init /etc/init.d/nims
\end{lstlisting}
will copy it to \shell{/etc/init.d} and rename it; this latter
step is important, in order for the \shell{service} command to
work as documented here.

For one-time activation on a new machine, use
\begin{lstlisting}
    /usr/sbin/update-rc.d nims start 99 2 3 4 5 . stop 1 0 1 6 .
\end{lstlisting}
This will set \shell{nims} to start in runlevels 2--5 at S99, 
and stop in runlevels 0, 1, and 6 at K1 (last to start, first
to stop in each runlevel).

For regular start/stop/restart of the \shell{nims} system, you can use
\begin{lstlisting}
    sudo service nims [start|stop|restart]
\end{lstlisting}
from the command prompt as usual for system services. Note that
the script must be run as root, but it will set each process to be
owned by the \shell{nims} unprivileged user.

\subsection{Philosophy}
Briefly, the reason for setting this up as a service is to allow for
operation as an embedded system. I think it's best to run our services
as unprivileged processes, though, in order to avoid potential security
issues. Since the system will be accessible remotely, we don't want
webapp users or people just looking at data to accidentally reconfigure
the network interface or disable the system.

This does require hardcoded paths, notably in the init script, and
in the config file. It also causes some issues with permissions if
you're not careful.

\subsection{Logging}
The \shell{nims} binaries log all output to \path{/var/tmp}
in a subfolder munged with the user's name and UID. For example,
user \shell{amaxwell} with UID 1000 has log files in
\path{/var/tmp/NIMS-amaxwell-1000/log}. This allows multiple users
to run the program without hitting permissions problems. Log
files are rotated periodically, and the latest log file has a
higher number appended. You can monitor these in the shell with
\begin{lstlisting}
    tail -f /var/tmp/NIMS-$USER-$UID/log/nims_0.log
\end{lstlisting}
for example.

\subsection{Networking}
The outer Ethernet port (near the edge of the box) is set for
DHCP, to be used on the PNNL or other managed networks. The
other port can be configured for a static IP, in order to be
used with instrument networks. To configure it for
a static ip, use \shell{sudo nm-connection-editor} if you
have an X11 display.

If you don't have X11, then you can try to
edit \path{/etc/NetworkManager/system-connections/Wired connection 2} 
or \path{/etc/NetworkManager/system-connections/Wired connection 1}. Check the
MAC adress in the connection file vs. the output of \shell{ifconfig -a} to
make sure you edit the right file, if you're doing this over ssh. If you
kill the port that you're connected on, you'll have to connect over the
other port or on the console.

For DHCP, the IPV4 section should look like this:
\begin{lstlisting}
    [ipv4]
    method=auto
    dhcp-client-id=nims1
\end{lstlisting}

For a static IP of 192.168.5.100, it should look like this:
\begin{lstlisting}
    method=manual
    address1=192.168.5.100/24,0.0.0.0
\end{lstlisting}
(i.e., a subnet mask of 255.255.255.0 and no gateway).
The supported way to modify those files is using \shell{nmcli}, but
it throws errors about mismatched versions and generally doesn't
work.


\end{document}


